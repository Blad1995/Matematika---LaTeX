\documentclass[12pt,a4paper]{article}
\usepackage[unicode,colorlinks=true]{hyperref}
%\usepackage[czech]{babel}
\usepackage[utf8]{inputenc}
\usepackage[T1]{fontenc}
\usepackage{lmodern}
\usepackage{graphicx}
\usepackage{amssymb}
\usepackage{amsmath}
\usepackage{amsfonts}
\usepackage{enumitem}
\textwidth 17.0cm \textheight 25.6cm
\topmargin -2.0 cm
\oddsidemargin 0cm
\hoffset -0.5cm

\pagestyle{plain}
\setcounter{page}{1}
\pagenumbering{arabic}

\begin{document}
	\title{\vspace{-1.5cm}Přehled MB204}
	\author{Jan Horáček; me@apophis.cz \\ Ondřej Divina; Ondra.Divina@seznam.cz}
	\maketitle
	
	\section{Kongruence a kryptografie}
	
	\begin{enumerate}[leftmargin=*]
		\item \textbf{Bezoutova věta}\\
		$\forall a_1,a_2 \in  \mathbb{Z} $ existuje $gcd(a_1,a_2) \land \exists k_1,k_2 \in  \mathbb{Z} $ takové, že~platí:
		\[(a_1,a_2) = k_1 \cdot a_1 + k_2 \cdot a_2\]
		jejich nalezení lze provést přes Euklidův algoritmus.
		
		\item \textbf{Dokonalá čísla}\\
		Čísla ve tvaru $\sigma(a) = 2a$, kde $\sigma(a)$ je součet dělitelů čísla $a$. Příkladem může být 6 nebo 28. 
		Sudé dokonalé čísla jsou ve tvaru $2^{q-1}\cdot(p^q - 1)$, kde $(p^q - 1)$ je Mersenneho prvočíslo. Neví se, zda existuje liché dokonalé číslo.
		
		\item \textbf{Mersenneho prvočísla}\\
		Prvočísla ve tvaru $2^k - 1$. Lze je jednoduše testovat pomocí Lucas-Lehmerova testu.
		\begin{itemize}
			\item  \textbf{Lucas-Lehmerův test}: posloupost $(s_n)_{n=0}^{\infty}$ kde $s_0 = 4; s_{n+1} = s_n^2 - 2$\\
			$M_p = 2^p - 1 $ je prvočíslo $\Leftrightarrow M_p | s_{p-2}$
		\end{itemize}
		
		\item \textbf{Počet prvočísel}\\
		Euklidova věta říká, že\textit{ „Mezi $\mathbb{N}$ existuje $\infty$ mnoho prvočísel“.}\\
		$\pi(x)$ udává počet prvočísel menších nebo rovných číslu $x \in \mathbb{R}$. \\
		\[ \pi(x) \sim \frac{x}{\ln{x}} \]
		Navíc \textbf{Čebyševova} (Betrandova) věta říká: \\\textit{„Pro libovolné číslo $n>1$ existuje alespoň jedno $p$ splňující $n<p<2n$“}
		
		%	\item Möbiova funkce.\\
		%		Rozložme přirozené číslo $n$ na prvočísla: $n = p_1^{\alpha_1} \cdots p_k^{\alpha_k}$.
		%		Hodnotu Möbiovy funkce $\mu(n)$ definujeme rovnu $0$, pokud pro nekteré
		%		$i$ platí $\alpha_i > 1$ a rovnu $(-1)^k$ v~opačném případě. Dále definujeme
		%		$\mu(1) = 1$.
		
		\item \textbf{Eulerova funkce}
		\[ \varphi(n) = |\{ a \in \mathbb{N} \mid 0 < a \le n, (a,n) = 1\}| \]
		Eulerova funkce v~bodě $n$ je počet přirozených čísel v~intervalu
		$\langle 1..n \rangle$, která jsou s~číslem~$n$ \textbf{nesoudělná }.
		
		$\varphi(1) = 1$, $\varphi(5) = 4$, $\varphi(6) = 2$, $\varphi(p) = p-1$.
		
		\[ \varphi(n) = n \cdot \left( 1 - \frac{1}{p_1} \right) \cdot
		\left( 1 - \frac{1}{p_2} \right) \cdots \left( 1 - \frac{1}{p_k} \right) \]
		
		kde $n = p_1\cdot p_2  \cdots  p_n$
		
		Funkce $\varphi$ je multiplikativní pro nesoudělná $a$, $b$:
		\[ \varphi(a \cdot b) = \varphi(a) \cdot \varphi(b). \]
		\[ \varphi(p^\alpha) = p^\alpha - p^{\alpha-1} = (p - 1) \cdot p^{\alpha-1}\]
		%		\[ \varphi(p_1^{\alpha_1} \cdots p_k^{\alpha_k}) = (p_1 - 1) \cdot
		%			p_1^{\alpha_1-1} \cdots (p_k - 1) \cdot p_k^{\alpha_k-1} \]
		%		\[ \varphi(n) = n \cdot \left( 1 - \frac{1}{q_1} \right) \cdots
		%			\left( 1 - \frac{1}{q_k} \right) \]
		
		\item \textbf{Malá Fermatova věta}\\
		Nechť $a \in \mathbb{Z}$, $p$ je prvočíslo a $p \nmid a$. Pak
		\[ a^{p-1} \equiv 1 \pmod{p} \]
		Věta poskytuje částečnou (= nezbytnou) podmínku prvočíselnosti.
		
		Napříkad pro \textit{Carmichaelova čísla} věta \textbf{platí}, ale přesto tato
		čísla nejsou prvočísla.
		
		\item \textbf{Eulerova věta}\\
		Nechť $a \in \mathbb{Z}$, $m \in \mathbb{N}$ a $(a,m) = 1$. Pak
		\[ a^{\varphi(m)} \equiv 1 \pmod{m} \]
		Je zobecněním Fermatovy věty.
		
		\item \textbf{Úplná soustava zbytků}\\
		Úplná soustava zbytků modulo $m$ je libovolná $m$-tice čísel po dvou
		nekongruentních modulo $m$. Nejčastěji $0, 1, \dots, m-1$, nebo
		(pro lichá $m$) $-\frac{m-1}{2}, \dots, -1, 0, 1, \dots, \frac{m-1}{2}$.
		
		\item \textbf{Redukovaná soustava zbytků}\\
		Redukovaná soustava zbytků modulo $m$ je libovolná $\varphi(m)$-tice
		čísel nesoudělných s~$m$ a po dvou nekongruentních modulo $m$.
		
		\item \textbf{Řád čísla modulo $m$}\\
		Nechť $a \in \mathbb{Z}$, $m \in \mathbb{N}$ a $(a,m) = 1$. Řádem čísla
		\textit{a modulo m} rozumíme nejmenší přirozené číslo $n$ splňující:
		\[ a^n \equiv 1 \pmod{m} \]
		Z toho vyplývá, že řád $ n $ je nutně $ \leq \varphi(n) $\\
		Pro  $a \in \mathbb{Z} \; \textnormal{řádu} $ řádu $ r$: \[ (a,n) = 1: a^t \equiv a^s \pmod{m} \Leftrightarrow t \equiv s \pmod{r} \] 
		
		\item \textbf{Primitivní kořen}\\
		Nechť $m \in \mathbb{N}$. Celé číslo $g \in \mathbb{Z}$, $(g,m) = 1$
		nezveme \textit{primitivním kořenem modulo m}, pokud je jeho řád modulo
		$m$ roven $\varphi(m)$.
		
		Pro lichá prvočísla \textbf{existují} primitivní kořeny.
		
		Umocňováním primitivního kořene modulo $p$ dostaneme všechny prvky
		\textbf{redukované} soustavy zbytků modulo $k$.
		
		\item \textbf{Wilsonova věta}\\
		Udává nutnou i postačující podmínku prvočíselnosti.
		
		Přirozené číslo $n > 1$ je prvočíslo, právě když
		\[ (n-1)! \equiv -1 \pmod{n}. \]
		Výpočetně extrémně náročné $\rightarrow$ nepoužívá se
		
		\pagebreak
		\item \textbf{Čínská zbytková věta}\\
		Nechť $m_1, \ldots, m_k \in \mathbb{N}$ jsou po dvou nesoudělná a
		$a_1, \ldots, a_k \in \mathbb{Z}$.
		
		Pak soustava
		\begin{align*}
		x &\equiv a_1 \pmod{m_1}\\
		&\vdots\\
		x &\equiv a_k \pmod{m_k}
		\end{align*}
		
		má jediné řešení modulo $m_1 \cdot m_2 \cdots m_k$.
		
		\item \textbf{Binomické kongruence}
		\[ x^n -a \equiv 0 \pmod{m} \]
		
		\begin{enumerate}
			\item \textbf{Mocninné zbytky}\\
			Nechť $m \in \mathbb{N}$, $a \in \mathbb{Z}$, $(a,m) = 1$. Číslo $a$
			nazveme \textit{n-tým mocninným zbytkem modulo m}, pokud je kongruence
			\[ x^n \equiv a \pmod{m} \]
			řešitelná. V opačném případě nazveme $a$ \textit{n-tým mocninným
				nezbytkem modulo m}.
			
			V~redukované soustavě zbytků modulo $m$ je \textbf{stejný} počet
			kvadratických zbytků a nezbytků.
			
			\item \textbf{Existence řešení binomických kongruencí}\\
			Buď $m \in \mathbb{N}$ takové, že modulo $m$ existují primitivní
			kořeny.  Dále nechť $a \in \mathbb{Z}$, $(a,m) = 1$. Pak kongruence
			$x^n \equiv a \pmod{m}$ je řešitelná (tj. $a$ je n-tý mocninný
			zbytek modulo $m$), právě když
			\[ a^{\varphi(m)/d} \equiv 1 \pmod{m} \]
			kde $d = (n, \varphi(m))$ je zároveň počet řešení.
		\end{enumerate}
		
		\item \textbf{Polynomiální kongruence}
		\begin{enumerate}
			\item \textbf{Henselovo lemma}\\
			udává postup pro řešení kongruencí modulo mocnina
			prvočísla.
			
			Nechť $p$ je prvočíslo, $f(x) \in \mathbb{Z}[x]$, $a \in
			\mathbb{Z}$ je takové, že $p \mid f(a)$, $p \nmid f'(a)$. Pak
			platí: pro každé $n \in \mathbb{N}$ má soustava
			\begin{align*}
			x &\equiv a \pmod{p}\\
			f(x) &\equiv 0 \pmod{p^n}
			\end{align*}
			právě jedno řešení modulo $p^n$. Viz příklad 4/10.
			
			\item \textbf{Řešitelnost kvadratických kongruencí}\\
			Podmínka řešitelnosti kongruence
			\[ ax^2 + bx + c \equiv 0 \pmod{m} \]
			je ekvivalentní podmínce řešitelnosti binomické kongruence
			\[ x^2 \equiv a \pmod{p}. \]
			
			\item \textbf{Legendreův symbol}\\
			Uvažujme kongruenci $x^2 \equiv a \pmod{p}$, která je řešitelná
			$\Leftrightarrow$ $a$ je kvadratický zbytek mod $p$. Zda-li je
			$a$ kv. zbytek určíme relativně snadno pomocí \textit{Legendreova
				symbolu}.
			Nechť $p$ je liché prvočíslo.
			
			\[ \genfrac(){}{0}{a}{p} = \left\{
			\begin{array}{ll}
			1 & p \nmid a, a \text{ je kvadratický zbytek modulo } p \\
			0 & p \mid a \\
			-1& p \nmid a, a \text{ je kvadratický nezbytek modulo } p
			\end{array}
			\right. \]
			
			Nechť $p$ je liché prvočíslo a $a, b \in \mathbb{Z}$ jsou libovolná.
			Pak platí:
			
			\begin{enumerate}
				\item $\genfrac(){}{1}{a}{p} \equiv  a^{\frac{p-1}{2}} \pmod{p},$
				\item $\genfrac(){}{1}{ab}{p} = \genfrac(){}{1}{a}{p} \cdot
				\genfrac(){}{1}{b}{p},$
				\item $a \equiv b \pmod{p} \Rightarrow \genfrac(){}{1}{a}{p}
				= \genfrac(){}{1}{b}{p}.$
			\end{enumerate}
			
			\item \textbf{Kvadratická reciprocita po Legendreův symbol}\\
			Nechť $p$, $q$ jsou lichá prvočísla. Pak:
			
			\begin{enumerate}
				\item $\genfrac(){}{1}{-1}{p} = (-1)^{\frac{p-1}{2}} \Leftrightarrow$
				$-1$ je kvadratický zbytek $\Leftrightarrow p \equiv 1 \pmod{4}$.
				\item $\genfrac(){}{1}{2}{p} = (-1)^{\frac{p^2-1}{8}} \Leftrightarrow$
				$2$ je kvadratický zbytek $\Leftrightarrow p \equiv \pm 1
				\pmod{8}$.
				\item $\genfrac(){}{1}{q}{p} = \genfrac(){}{1}{p}{q} \cdot
				(-1)^{\frac{p-1}{2} \cdot \frac{q-1}{2}} \Leftrightarrow$
				znaménko se nemění $\Leftrightarrow p \equiv 1 \vee
				q \equiv 1 \pmod{4}$.
			\end{enumerate}
			
			
			\item \textbf{Jacobiho symbol}\\
			Legendre platí jen pro lichá prvočísla.  Tento problém řešíme zavedením \textit{Jacobiho
				symbolu}.
			
			Nechť $a \in \mathbb{Z}$, $b \in \mathbb{N}$, $2 \nmid b$. Nechť
			$b = p_1 \cdot p_2 \cdots p_k$ je rozklad čísla $b$ na (lichá)
			prvočísla (neskupujeme stejná prvočísla do mocniny!). Symbol
			\[ \genfrac(){}{0}{a}{b} = \genfrac(){}{0}{a}{p_1} \cdot
			\genfrac(){}{0}{a}{p_2} \cdots \genfrac(){}{0}{a}{p_k} \]
			
			se nazývá \textit{Jacobiho symbol}.
			
			Pozor: pro \textit{Jacobiho symbol} neplatí implikace
			\[ (a/b) = 1 \Rightarrow x^2 \equiv a \pmod{b} \text{ je řešitelná.} \]
			
			Kvadratická reciprocita stejná jako pro \textit{Legendreův symbol}.
			
			\item \textbf{Gaussovo lemma}\\
			Pomocí funkce $\mu$ lze snadno spočítat hodnotu \textit{Legendreova
				symbolu}.
			
			\[ \genfrac(){}{0}{a}{p} = (-1)^{\mu(a)} \]
			
			$\mu(a)$ je počet záporných prvků $s$, pro které platí.
			
			\centering 
			$S = \left\{1, 2, \dots, \frac{p-1}{2}\right\}$,
			$\forall x \in S : a \cdot x \equiv \pm s \in S$. 
			
		\end{enumerate}
		
		
		
		\pagebreak	
		\item \textbf{Sifrovací systémy (algoritmy)}
		\begin{enumerate}
			\item \textbf{Rabinův kryptosystém}\\
			Dokázáno, že je k prolomení potřeba faktorizovat modul. (první, u~kterého to bylo dokázáno).
			
			\begin{enumerate}
				\item $p,q \equiv 3 \pmod{4}$
				\item $n = p \cdot q$
				\item $V_A = n~;~S_A = (p,q)$
				\item Zašifrování zprávy: $C \equiv M^2 \pmod{n}$
				\item Dešifrování jako výpočet 4 odmocnin z $C \pmod{n}$. Jedna z nich je původní zprávou
			\end{enumerate}
			
			\item \textbf{RSA}\\
			Těžké naprogramovat bez postranních kanálů. Bezpečnost je zaručena při použití velkého $d$ a zároveň $ p$ a $q$ si nesmí být příliš blízká.
			\begin{enumerate}
				\item Zvolí se velké prvočísla $p,q$
				\item $n = p \cdot q~;~ \varphi(n) = (p-1)\cdot(q-1)$
				\item Zvolí se libovolný veřejný klíč $e$, pro který platí $(e,\varphi(n)) = 1$
				\item Spočítá se \textbf {tajný klíč}  $d$ tak, že $e\cdot d \equiv 1 \pmod{\varphi(n)}$
				\item \textbf{Šifrování:} $M \rightarrow C=C_e(M) \equiv M^e \pmod{n}$
				\item \textbf{Dešifrování}  (C je šifrovaný text): $C^d \pmod{n} = (M^e)^d \equiv M \pmod{n}$
			\end{enumerate}
			
			
			\item \textbf{Diffie-Hellman výměna klíčů}\\
			Výměna klíčů pro symetrickou kryptografii. Založeno na problému \textbf{diskrétního~logaritmu} (DLP). potřeba \textbf{autentizace} (\textit{man in the middle attack})
			\begin{enumerate}
				\item Veřejně se strany dohodnou na \textbf{prvočísle} a primitivním kořenu $g \pmod{p}$
				\item Alice vybere $a$ a pošle $g^a \pmod{p}$~;~Bob vybere $b$ a pošle $g^b \pmod{p}$
				\item Společným klíčem je $g^{ab} \pmod{p}$
			\end{enumerate}
			
			
			\item \textbf{ElGamal}\\
			Podobný Diffie-Hellman. Lze použít i na podepisování (jako RSA).
			\begin{enumerate}
				\item Alice zvolí prvočíslo $p$ a primitivní kořen $g$
				\item Alice zvolí \textbf{tajný klíč} $x$, spočítá $h=g^x \pmod{p}$~;~ $V_A = (p,g,h)$
				\item Bob zvolí $y$ a vypočte $C_1 = g^y \pmod{p}$ a $C_2 = M \cdot h^y \pmod{p}$;~a~pošle $ (C_1,C_2)$
				\item Dešifrování: $M =\frac{C_2}{(C_1)^x}$;~kde $C_1^x = h^y =g^{xy}$
			\end{enumerate}
		\end{enumerate}
		
	\end{enumerate}
	
	\pagebreak
	\section{Algebra - Teorie grup}
	
	\begin{enumerate}
		\item \textbf{Struktury s jednou operací}\\
		\begin{enumerate}
			\item Grupoid - uzavřená na binární operaci $\cdot$
			\item Pologrupa - asociativní 
			\item Monoid - pologrupa s neutrálním prvkem
			\item Grupa - monoid, který obsahuje inverzi pro každý prvek
			\item Komutativní (cokoliv) - operace $\cdot $ je komutativní (abelovská grupa)
		\end{enumerate}
		
		\item \textbf{Parita permutace}\\
		Každý permutace (konečné) množiny je složením cyklů. Cyklus délky $l$ lze složit z $l-1$ transpozic. $\rightarrow$ Parita cyklu délky $l$ je $(-1)^{l-1}$\\
		Parita složení permutací je součinem parit jednotlivých z nich. Tzn. zobrazení $sgn$ převádí permutaci $\sigma \circ \pi$ na součin  sgn $\sigma$ $ \cdot$ sgn $\pi$	
		
		\item \textbf{Dihedrální grupy}\\
		Grupy symetrií s $k$ různýmmi rotacemi a $k$ různými zrcadleními. Např. pravidelný k-úhelník. Označují se jako $D_{2k}$. Grupy řádu 2k (někdy jenom D(k)). Pro $k  \geq 3$
		
		\item \textbf{Cyklické grupy}\\
		Obrazce, které mají pouze rotační symetrie se (např v chemii) značí $C_k$. Cyklické grupy řádu $k$.
		Necht’ je M ohraničená množina v rovině $\Re^2$. Pak grupa jejich symetrií je bud’ triviální nebo jedna z grup $C_k , D_{2k} $, kde$~k \geq 1$.
		\item \textbf{Pologrupy a podpologrupy}\\
		Je-li $(A,\cdot)$ grupa (pologrupa), pak její podmnožina uzavřená na zúžení operace $\cdot$ a zároveň je s touto operací \textbf{podgrupou} (podpologrupou).
		
		\item \textbf{Homomorfismus}\\
		Zobrazení 	$f~:~(G,\cdot) \rightarrow (H,\circ)$ mezi dvěma grupami se nazývá  \textbf{Homomorfismus grup}, pokud respektuje násobení tak, že $\forall a,b \in G;~ f(a\cdot b) = f(a) \circ f(b)$ \\Bijektivní homomorfismus se nazývá \textbf{Izomorfismus}. Značíme $G \cong H$ \\ \textit{Pro každý homomorfismus $f:~G \rightarrow H$ platí:}
		\begin{enumerate}
			\item Obraz neutrálního prvku $e_G \in G$ je neutrální prvek v H.
			\item Obraz inverze k prvku je inverzí obrazu. Tj. $f(a^{-1}) = f^{-1}(a)$
			\item Obraz podgrupy $K \subset G$ je podgrupa $f(K) \subset H$
			\item Vzorem $f^{-1}(K) \subset G pogrupy K \subset H$ je podgrupa.
			\item Je-li f zároveň bijekcí, pak i inverzní zobrazení $f^{-1}$ je homomorfismus
			\item f je injektivní zobrazení $\Leftrightarrow f^{-1}(e_H)= \{e_G\}$
		\end{enumerate}
		Podgrupa, která je vzorem jednotkového prvku $e \in H$ se nazývá \textbf{jádro} homomorfismu $f$ a značíme ji $ker f$
		
		\item \textbf{(Přímý) součin grup}\\
		Pro každé dvě grupy $(G,\cdot), (H,\circ)$ definujeme \textbf{součin grup} $(G \times H, \star)$ takto:\\ $G \times H$ je kartézský osučin, kde grupové násobení je definováno po složkách
		$(a,x) \star (b,y) = (a \cdot b, x \circ y)$
		\begin{enumerate}
			\item \textit{Cyklické grupy} \\
			Libovolný prvek \textit{a} v grupě  G je obsažen v minimální podgrupě\\ $\{e = a^0, a = a^1, a^2, \dots\}$, která jej obsahuje. Nutně je tato podgrupa \textbf{komutativní} a musí nastat případ $a^k = e$
			Nejmenší takové $k$ se nazývá \textbf{řád prvku} $a$ v G\\ Grupa G je \textbf{cyklická}, je-li celé G generované nějakým svým prvkem $a$ výše uvedeným způsobem.\\ 
			Pozn.: Každá cyklická grupa je izomorfní buď $\mathbb{Z}$ pokud je nekonečná a nebo některé grupě zbytkových tříd $(\mathbb{Z}_k,+)$
		\end{enumerate}
		
		\item \textbf{Levé a pravé třídy rozkladu} \\
		Celá grupa G se rozpadá na tzv. \textbf{levé třídy rozkladu} podle podgrupy H vzájemně ekvivalenstních prvků.  Pro třídy příslušející prvku (značíme a $\cdot H$) platí: $a \cdot H = \{a \cdot h; h \in H\}$. G/H - množina všech levých tříd rozkladu podle podgrupy H
		Obdobně definujeme $H \cdot a$ jako pravou třídu rozkladu, $H \backslash G = \{H \cdot a; a \in G\}$.
		
		\begin{enumerate}
			\item Levé a pravé třídy rozkladu podle podgrupy H $\leq$ G plývají právě tehdy, když pro každé $a \in G, h \in H$ platí $a \cdot h \cdot a^{-1} \in H$.
			\item Všechny třídy (levé i pravé) mají shodnou mohutnost jako podgrupa H.
			\item Zobrazení $a \cdot H \mapsto H \cdot a^{-1}$ zadává bijekci mezi levými a pravými třídami rozkladu G podle H.
		\end{enumerate}
	
	\item \textbf{Pogrupy}
		Nechť G je konečná grupa s n prvky (G je řádu n), H její \textbf{podgrupa}.
		\begin{enumerate}
			\item Mohhutnost n = |G| je součinem mohutnosti H a mohutnosti G/H. |G| = |G/H|.|H|.
			\item Přirozené číslo |H| je dělitelem čísla n.
			\item Je-li a $\in$ G prvek řádu k, pak k dělí n.
			\item Pro každé a $\in$ G je $a^n = e$.
			\item Je-li mohuntost grupy G prvočíslo p, pak je G izomorfní cyklické grupě $\mathbb{Z}_p$. 
		\end{enumerate}
	
		\textbf{Podgrupa H je normální} právě tehdy, když pro každé $a \in G$ ptatí $a\cdot H = H\cdot a$ (levý rozklad G podle podgrupy H je shodný s pravým rozkladem).
		\begin{enumerate}
			\item 1 $\lhd$ G, G $\lhd$ G
			\item V komutativní grupě je každá podgrupa normální.
			\item Je-li H podgrupa konečné grupy G, kde $|H|=|G|/2$, pak je H normální.
		\end{enumerate}
	
	\item \textbf{Hlavní věta konečných komutativních grup}
		Je-li G konečná komutativní grupa, pak platí: $G\cong \mathbb{Z}_{n1} \times \mathbb{Z}_{n1}\cdots\times\mathbb{Z}_{nk}$, pro vhodná $n_1,\cdots,n_k\in\mathbb{N}\backslash\{1\}$ splňující $n_{i+1}|n_i$ pro $1 \leq i \leq k-1$. Tento rozklad je přitom jednoznačný.
		\begin{enumerate}
			\item Každé provočíslo p, dělící řád grupy G, dělí $n_1$.
			\item Je-li n součinem různých prvočísel, pak jedinou cyklickou komutativní grupou řádu n je (až na isomorfismus) cyklická grupa $\mathbb{Z}_n$.
		\end{enumerate}
		
		\textbf{Cauchyho věta} \\
		Je-li konečná grupa G řádu dělitelného prvočíslem p, pak obsahuje prvek řádu p.\\ \\
		\textbf{Sylowova věta} \\
		Konečná grupa G řádu p$^\alpha$m (p je prvočíslo, p $\nmid$ m) má vždy podgrupu řádu p$^\alpha$. Pro počet n$_p$ takových podgrup platí n$_p \equiv 1 (mod p)$ a n$_p$ | m. 
		
	\item \textbf{Jordan-Hölderova věta a kompoziční řada} \\
	Posloupnost podgrup grupy G $\{e\} = N_0 \leq N_1 \leq \cdots \leq N_k = G$ se nazývá kompoziční řada, pokud $N_i \lhd N_{i+1} a N_{i1+}/N_i$ je prostá.  \\Konečná grupa má vždy kompoziční řadu, která je jednoznačně určená až na izomorfismus faktorů (tj. počet členů dvou takových řad je stejný a příslušné faktorgrupy jsou izomorfní).
	
	\item \textbf{Věty o izomorfismu} \\
	Pro libovolný homomorfismus grup f : G $\rightarrow$ K je dobře definován také homomorfismus $\tilde{f} : G/ker f \rightarrow K, \tilde{f}(a \cdot ker f) = f(a)$, který je injektivní. Zejména dostáváme $G/ker f \cong$ f(G). \\
	
	Nechť A, B $\leq$ G jsou podgrupy splňující $A \leq N_G(B)$. Pak (A $\cap$ B) $\lhd$ A a platí AB/B $\cong$ A/(A $\cup$ B). \\
	
	Jsou-li $A, B\ \lhd \ G$ normální podgrupy splňující $A \leq B$, pak $B/A \lhd G/A$ a platí $(G/A)/(B/A) \cong G/B$. \\
	
	Nechť je $N \ \lhd\ G$. Pak ewistuje bijekce mezi množinou podgrup A obsahujících N a množinou podgrup A/N. Navíc normálním podgrupám odpovídají normální podgrupy.
	
	\item \textbf{Akce grupy}
	
	\begin{enumerate}
		\item \textbf{Levá akce grupy} G na množině X je homomorfismus grupy G do podgrupy invertibilních prvků v podgrupě X$^X$ všech zobrazení X $\rightarrow$ X, které splňuje 
		\begin{align*}
		\varphi(a\cdot b,x) = \varphi(a,\varphi(b,x)).\	\ \ \ \	(a\cdot b)\cdot x = a\cdot (b\cdot x)
		\end{align*} 
		
		\item Obraz prvku x $\in$ X v akci celé grupy G nazýváme \textbf{orbita} X$_X$ prvku x, tj.
		\begin{align*}
		X_X = \{y = \varphi(a,x); a\in G\}.
		\end{align*} 
		
		\item Pro každý bod x$\in$X definujeme \textbf{izotropní podgrupu} (též stabilizátor) $G_X \subseteq G$ akce $\varphi$:
		\begin{align*}
		G_X = \{a \in G;\ \varphi(a,x) =x\}.
		\end{align*}
		\item 
		Jestliže pro každé dva prvky x, y $\in$ X existuje a $\in$ G tak, že $\varphi$(a,x) = y, pak říkáme, že akce $\varphi$ je \textbf{tranzitivní}. \\
		např. Přirozenou akci na množině levých tříd G/H pro jakoukoliv podgrupu H definujeme vztahem $g\cdot (aH) = (ga) H$.
		
		\item Pro každou akci koknečné grupy G na konečné množině X platí:
		\begin{enumerate}
			\item pro každý prvek x $\in$ X je $|G| = |G_X|\cdot |X_X|$,
			\item (Burnsideovo lemma) je-li N počet orbit akce G na X, pak $|G| = \frac{1}{N}\sum_{g\in G}|X^g|$, kde $X^g = \{x \in X; g\cdot x = x\}$ označujeme množinu pevných bodů akce prvku g.
		\end{enumerate}
	\end{enumerate}

	\item \textbf{Okruhy a tělesa}
	\begin{enumerate}
		\item Komutativní grupa (R,+) s neutrálním prvkem 0 $\in$ R, spolu s další operací $\cdot$ splňující
		 \begin{enumerate}
		 	\item (a $\cdot$ b) $\cdot$ c = a $\cdot$ (b $\cdot$ c), pro všchna a, b, c $\in$ R (asociativita)
		 	\item a $\cdot$ b = b $\cdot$ a pro všechna a, b $\in$ R (komutativita)
		 	\item existuje prvek 1 takový, že pro všechny a $\in$ R platí 1 $\cdot$ a = a (existence jedničky)
		 	\item a $\cdot$ (b + c) = a $\cdot$ b + a $\cdot$ c, pro všechny a, b, c $\in$ R (distributivita)
		 \end{enumerate}
	 	se nazývá \textbf{komutativní okruh}. Takový okruh zapisujeme (R, +,$\cdot$).
	 	
	 	\item Jestliže v komutativním okruhuh R platí c $\cdot$ d = 0 právě, když je alespoň jeden z prvků c a d nulový, pak okruh R nazýváme \textbf{oborem integrity}.
	 	
	 	\item Platí-li v oboru integrity $a=b\cdot c$ a $b\neq0$, pak $c$ je jednoznačně dáno volbou $a,b$.
	\end{enumerate}
	
	
	
	
	
	
		
		
	\end{enumerate}
	
	
\end{document}
